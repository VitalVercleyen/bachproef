%%=============================================================================
%% Voorwoord
%%=============================================================================

\chapter*{\IfLanguageName{dutch}{Woord vooraf}{Preface}}
\label{ch:voorwoord}


Voor u ligt een onderzoek naar het herkennen van schriftsystemen door middel van deep learning.
Dit is geschreven in het kader van de afstudeerrichting Toegepaste Informatica aan de Hoge School Gent.
Van februari 2019 tot en met mei 2019 ben ik bezig geweest met het onderzoek en het schrijven van mijn bachelorproef.

Samen met mijn promotor, Steven Van Impe, heb ik de onderzoeksvraag voor deze bachelorproef bedacht.
Ik had veel baat aan de technische hulp die ik kreeg van Glenn Van Looveren, hij beantwoordde steeds mijn vragen waardoor ik steeds verder kon met mijn onderzoek.

Bij dezen wil ik graag mijn begeleiders bedanken voor de fijne begeleiding en hun ondersteuning tijdens dit traject. Ook wil ik alle respondenten bedanken die mee hebben gewerkt aan dit onderzoek. Zonder hun medewerking had ik dit onderzoek nooit kunnen voltooien.

Tevens wil ik ook mijn stagecollega Joris Opsommer bedanken die mij vaak heeft bijgestaan bij dit onderzoek.
Bovendien ben ik ook dankbaar voor de wijze raad van mijn vrienden en familie en met nadruk op mijn vriendin Aya Coppens die mij elke dag motivatie en inspiratie gaf.

Tot slot wil ik zeker mijn ouders bedanken die ervoor zorgden dat ik aan mijn studie kon beginnen en die zal afmaken.

Ik wens u veel leesplezier toe.

Vercleyen Vital
%% TODO:
%% Het voorwoord is het enige deel van de bachelorproef waar je vanuit je
%% eigen standpunt (``ik-vorm'') mag schrijven. Je kan hier bv. motiveren
%% waarom jij het onderwerp wil bespreken.
%% Vergeet ook niet te bedanken wie je geholpen/gesteund/... heeft


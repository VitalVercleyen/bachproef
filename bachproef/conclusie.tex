%%=============================================================================
%% Conclusie
%%=============================================================================

\chapter{Conclusie}
\label{ch:conclusie}

% TODO: Trek een duidelijke conclusie, in de vorm van een antwoord op de
% onderzoeksvra(a)g(en). Wat was jouw bijdrage aan het onderzoeksdomein en
% hoe biedt dit meerwaarde aan het vakgebied/doelgroep? 
% Reflecteer kritisch over het resultaat. In Engelse teksten wordt deze sectie
% ``Discussion'' genoemd. Had je deze uitkomst verwacht? Zijn er zaken die nog
% niet duidelijk zijn?
% Heeft het onderzoek geleid tot nieuwe vragen die uitnodigen tot verder 
%onderzoek?

Dit onderzoek ging na of het mogelijk is om schriftsystemen te onderscheiden met behulp van een convolutioneel neuraal netwerk.
De resultaten van het gemaakte model waren positief, het is gelukt om drie verschillende schriftsysteem van elkaar te onderscheiden met behulp van een convolutioneel neuraal netwerk.

Er werd verwacht dat het model deze drie schriftsystemen met niet veel moeite zou kunnen onderscheiden en dit met een hoge accuraatheid.
Dit was inderdaad het geval, wanneer de generaties waren voltooid werd er een test uitgevoerd op het model met ongeziene data en de accuraatheid was nog altijd hoog.

Het onderzochte levert een bijdrage aan het classificeren van individuele karakters onder hun schriftsysteem, dit kan gebruikt worden in het analyseren en lokaliseren van documenten met onbekende schriftsystemen.\\
Ook kan dit gebruikt worden vooraleer karakters van een specifiek schriftsystemen in een onbekend document moeten worden herkend. Eerst wordt bepaalt welk schriftsysteem er gebruikt wordt in het document en vervolgens kan er een andere programma de karakters herkennen onder het bepaalde schriftsysteem.

De drie gebruikte schriftsystemen staan elk onder een verschillende overkoepelende groep waardoor ze elk heel weinig gelijkende karakteristieken bevatten. \\
Naast het onderscheiden van deze drie schriftsystemen is er geëxperimenteerd met meerdere schriftsystemen aan het model aan te leren.\\
Een daarvan is het Devanagari, een systeem dat gebruikt wordt in Indië.
Dit schriftsysteem is zeer gelijkaardig aan het Kanji aangezien het met dezelfde hoeken, vormen en lijnen werkt.\\
Wanneer een dataset bestaande uit verschillende karakters van het Devanagari aan het model werd aangeleerd was er een lagere accuraatheid, ook wanneer het getest werd in de praktijk maakte het model fouten bij het herkennen van het Kanji of Devanagari aangezien het de systemen niet goed genoeg van elkaar kon onderscheiden.

Uit dit experiment kan er besloten worden dat als er software zou worden geschreven om vele verschillende schriftsystemen van elkaar te onderscheiden er een meer accuraat model geschreven moet worden met meer aandacht aan de gebruikte filters, het aantal neuronen en het aantal lagen die in het model worden gebruikt.
Wanneer dit niet genoeg zou zijn en er nog altijd een lage accuraatheid is zal er moeten gekeken worden naar het gebruik van andere technologieën dit het proces van het model kunnen vergemakkelijken.

Een grote factor die in acht moeten worden genomen is de beschikbaarheid van de nodige datasets. Een groot deel van de tijd dat in dit onderzoek werd gestoken in het vinden van goede datasets, wanneer dit zou gebeuren voor een groot aantal schriftsystemen is dit niet evident.








%%=============================================================================
%% Samenvatting
%%=============================================================================

% TODO: De "abstract" of samenvatting is een kernachtige (~ 1 blz. voor een
% thesis) synthese van het document.
%
% Deze aspecten moeten zeker aan bod komen:
% - Context: waarom is dit werk belangrijk?
% - Nood: waarom moest dit onderzocht worden?
% - Taak: wat heb je precies gedaan?
% - Object: wat staat in dit document geschreven?
% - Resultaat: wat was het resultaat?
% - Conclusie: wat is/zijn de belangrijkste conclusie(s)?
% - Perspectief: blijven er nog vragen open die in de toekomst nog kunnen
%    onderzocht worden? Wat is een mogelijk vervolg voor jouw onderzoek?
%
% LET OP! Een samenvatting is GEEN voorwoord!

%%---------- Nederlandse samenvatting -----------------------------------------
%
% TODO: Als je je bachelorproef in het Engels schrijft, moet je eerst een
% Nederlandse samenvatting invoegen. Haal daarvoor onderstaande code uit
% commentaar.
% Wie zijn bachelorproef in het Nederlands schrijft, kan dit negeren, de inhoud
% wordt niet in het document ingevoegd.

\IfLanguageName{english}{%
\selectlanguage{dutch}
\chapter*{Samenvatting}



\selectlanguage{english}
}{}

%%---------- Samenvatting -----------------------------------------------------
% De samenvatting in de hoofdtaal van het document

\chapter*{\IfLanguageName{dutch}{Samenvatting}{Abstract}}

Karakterherkenning is een veel besproken onderwerp binnen artificiële intelligentie, hierin gaat er veel aandacht naar het herkennen van karakters in een specifiek schriftsysteem.
Wat dit onderzoek trachtte te bereiken is het maken van een convolutioneel neuraal netwerk dat verschillende schriftsystemen kan onderscheiden.
Dit kan een voordeel bieden aan letterkundigen die een opdracht krijgen binnenin de forensische letterkunde om een bepaald document met een ongekend schriftsysteem te herkennen en te lokaliseren.

Het onderscheiden van schriftsystemen met behulp van artificiële intelligentie is al onderzocht maar dit steeds op basis van woorden en paragrafen.
Dit onderzoek pakt het anders aan, het gebruikt een convolutioneel neuraal netwerk om schriftsystemen te herkennen op basis van een individueel karakter, dit bied een voordeel aangezien individuele karakters vaak voorkomen in documenten wanneer de grootste inhoud van het document geschreven is in een ander schriftsysteem.

De inhoud van dit onderzoek bestaat uit een stand van zaken waarin de gebruikte technologieën worden besproken samen met een bespreking van het al onderzochte in dit onderzoek. \\
Vervolgens is er een methodologie uitgeschreven waarin er wordt uitgelegd hoe er te werk is gegaan bij het verzamelen van data, het ontwerpen en trainen van het convolutioneel neuraal netwerk en het gebruik van het geschreven model in de praktijk. \\
Daarnaast worden de resultaten besproken die aantonen dat het model drie verschillende schriften van elkaar kan onderscheiden met een hoge accuraatheid. \\
Uiteindelijk is er een conclusie die deze resultaten bevestigd en waarin er wordt besproken dat wanneer er getracht zou worden om meerdere schriftsystemen van elkaar te onderscheiden er meer tijd en werk zou moeten gestoken worden in het ontwerp van het model en de keuze van de gebruikte technologieën.